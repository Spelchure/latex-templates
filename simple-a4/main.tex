\documentclass[12pt, a4paper]{article}

% Use UTF-8 encoding
\usepackage[utf8]{inputenc}

% For images
\usepackage{graphicx}

% Code listings
\usepackage{minted}
 

% Images path
\graphicspath{{./img/}}

%For defining colors.
%\usepackage{xcolor}	

% Hyperlinks
%\usepackage{hyperref}

% Example color definitions (xcolor)
%\definecolor{codeBackgroundColor}{rgb}{0.95, 0.95, 0.92}
%\definecolor{codeGray}{rgb}{0.5,0.5,0.5}

% Hyper package setup
%\hypersetup{
%	colorlinks=true,
%	linkcolor=blue,
%	urlcolor=cyan
%}

% Renewing commands
%\renewcommand{\partname}{NewPartName}
%\renewcommand{\figurename}{NewFigureName}

\begin{document}
  
\part{Simple Part}

\section{Simple Section}

\subsection{Minted Usage}

\begin{minted}{python}
import os
import sys

def main():
    print("hello world")

if __name__ == '__main__':
    main()
\end{minted}

This is single line \mint{python}|import this| source code.

This is inline \mintinline{python}{print(x**2)} source code.

For code from file use these:
\begin{verbatim}
\inputminted[options]{language}{filename}
\end{verbatim} 

\subsection{Minted with Frame}

\begin{minted}[
frame=lines,
framesep=2mm,
baselinestretch=1.2,
fontsize=\footnotesize,
linenos
]
{python}
import numpy as np
    
def incmatrix(genl1,genl2):
    m = len(genl1)
    n = len(genl2)
    M = None #to become the incidence matrix
    VT = np.zeros((n*m,1), int)  #dummy variable
    
    #compute the bitwise xor matrix
    M1 = bitxormatrix(genl1)
    M2 = np.triu(bitxormatrix(genl2),1) 

\end{minted}

\end{document}